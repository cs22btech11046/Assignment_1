

\let\negmedspace\undefined
\let\negthickspace\undefined

\documentclass[journal,12pt,twocolumn]{IEEEtran}

%\documentclass[conference]{IEEEtran}
%\IEEEoverridecommandlockouts
% The preceding line is only needed to identify funding in the first footnote. If that is unneeded, please comment it out.
\usepackage{cite}
\usepackage{amsmath,amssymb,amsfonts,amsthm}
\usepackage{algorithmic}
\usepackage{graphicx}
\usepackage{textcomp}
\usepackage{xcolor}
\usepackage{txfonts}
\usepackage{listings}
\usepackage{enumitem}
\usepackage{mathtools}
\usepackage{gensymb}
\usepackage[breaklinks=true]{hyperref}
\usepackage{tkz-euclide} % loads  TikZ and tkz-base
\usepackage{listings}

                                           
\DeclareMathOperator*{\Res}{Res}
%\renewcommand{\baselinestretch}{2}
\renewcommand\thesection{\arabic{section}}
\renewcommand\thesubsection{\thesection.\arabic{subsection}}
\renewcommand\thesubsubsection{\thesubsection.\arabic{subsubsection}}

\renewcommand\thesectiondis{\arabic{section}}
\renewcommand\thesubsectiondis{\thesectiondis.\arabic{subsection}}
\renewcommand\thesubsubsectiondis{\thesubsectiondis.\arabic{subsubsection}}

% correct bad hyphenation here
\hyphenation{op-tical net-works semi-conduc-tor}
\def\inputGnumericTable{}                                 %%

\lstset{
%language=C,
frame=single, 
breaklines=true,
columns=fullflexible
}

\begin{document}
%

\newtheorem{theorem}{Theorem}[section]
\newtheorem{problem}{Problem}
\newtheorem{proposition}{Proposition}[section]
\newtheorem{lemma}{Lemma}[section]
\newtheorem{corollary}[theorem]{Corollary}
\newtheorem{example}{Example}[section]
\newtheorem{definition}[problem]{Definition}

%\newcommand{\BEQA}{\begin{eqnarray}}
%\newcommand{\EEQA}{\end{eqnarray}}
%\newcommand{\define}{\stackrel{\triangle}{=}}

\bibliographystyle{IEEEtran}


\newcommand{\define}{\stackrel{\triangle}{=}}
\newcommand{\permcomb}[4][0mu]{{{}^{#3}\mkern#1#2_{#4}}}
\newcommand{\comb}[1][-1mu]{\permcomb[#1]{C}}

%\bibliographystyle{ieeetr}

\providecommand{\mbf}{\mathbf}
\providecommand{\pr}[1]{\ensuremath{\Pr\left(#1\right)}}
\providecommand{\qfunc}[1]{\ensuremath{Q\left(#1\right)}}
\providecommand{\sbrak}[1]{\ensuremath{{}\left[#1\right]}}
\providecommand{\lsbrak}[1]{\ensuremath{{}\left[#1\right.}}
\providecommand{\rsbrak}[1]{\ensuremath{{}\left.#1\right]}}
\providecommand{\brak}[1]{\ensuremath{\left(#1\right)}}
\providecommand{\lbrak}[1]{\ensuremath{\left(#1\right.}}
\providecommand{\rbrak}[1]{\ensuremath{\left.#1\right)}}
\providecommand{\cbrak}[1]{\ensuremath{\left\{#1\right\}}}
\providecommand{\lcbrak}[1]{\ensuremath{\left\{#1\right.}}
\providecommand{\rcbrak}[1]{\ensuremath{\left.#1\right\}}}
\theoremstyle{remark}
\newtheorem{rem}{Remark}
\newcommand{\sgn}{\mathop{\mathrm{sgn}}}
\providecommand{\abs}[1]{\left\vert#1\right\vert}
\providecommand{\res}[1]{\Res\displaylimits_{#1}} 
\providecommand{\norm}[1]{\left\lVert#1\right\rVert}
%\providecommand{\norm}[1]{\lVert#1\rVert}
\providecommand{\mtx}[1]{\mathbf{#1}}
\providecommand{\mean}[1]{E\left[ #1 \right]}
\providecommand{\fourier}{\overset{\mathcal{F}}{ \rightleftharpoons}}
%\providecommand{\hilbert}{\overset{\mathcal{H}}{ \rightleftharpoons}}
\providecommand{\system}{\overset{\mathcal{H}}{ \longleftrightarrow}}
	%\newcommand{\solution}[2]{\textbf{Solution:}{#1}}
\newcommand{\solution}{\noindent \textbf{Solution: }}
\newcommand{\cosec}{\,\text{cosec}\,}
\providecommand{\dec}[2]{\ensuremath{\overset{#1}{\underset{#2}{\gtrless}}}}
\newcommand{\myvec}[1]{\ensuremath{\begin{pmatrix}#1\end{pmatrix}}}
\newcommand{\mydet}[1]{\ensuremath{\begin{vmatrix}#1\end{vmatrix}}}

\let\vec\mathbf


\vspace{3cm}

\title{
\textbf{Assignment 1}\\ \textbf{AI1110}: Probability and Random Variables\\Indian Institute of Technology Hyderabad
	}

\author{CS22BTECH11046\\P.Yasaswini}	
 




\maketitle

\newpage

%\tableofcontents

\bigskip

\renewcommand{\thefigure}{\theenumi}
\renewcommand{\thetable}{\theenumi}



\textbf{10.15.1.25: Question}. Which of the following arguments are correct and which are not correct?Give reasons for your answer.
\begin{enumerate}
\item If two coins are tossed simultaneously, there are three possible outcomes - two heads, two tails, or one of each. Therefore, for each of these outcomes, the probability is $\frac{1}{3}$.

\item If a die is thrown, there are two possible outcomes - an odd number or an even number. Therefore, the probability of getting an odd number is $\frac{1}{2}$.
\end{enumerate}
\textbf{Solution}:
\begin{enumerate}
\item X is a random variable which denotes the number of heads obtained when n coins are tossed simultaneously,p = probability of getting head .\\


\begin{align}
X\sim \text{Bin}(n,p)
\end{align} 
then,
\begin{align}
	p_X(r) = \comb{n}{r} p^r (1-p)^{n-r}
\end{align}
Here n=2 and p=$\frac{1}{2}$, \\
Therefore,
\begin{align}
       p_X(1) &= \comb{2}{1} \times \frac{1}{2^2}
\end{align}
\textbf{Reason:} For X=1, it contains two mutually exclusive events (H, T), (T, H). Therefore, the probability of getting one of each is $\frac{1}{2}$ and not $\frac{1}{3}$. So, the above statement is incorrect.

\item Here, let E be the event ‘getting an odd number’.
\begin{align}
\text{Sample space }\Omega &= \{1,2,3,4,5,6\} \\
\text{E } &= \{1,3,5\} \\
	\pr{E} &= \frac{3}{6} = \frac{1}{2}
\end{align}

\textbf{Reason:} Event of getting an odd number and Event of getting an even number are equally likely and they together forms an exhaustive event.
Hence,
\begin{align}
	\pr{E} &= \frac{1}{2}\\
	\pr{E^\prime} &= \frac{1}{2}
\end{align}
\\So the above statement is correct.
\end{enumerate}
\end{document}


